\documentclass[10pt,a4paper,danish]{article}
%% Indlæs ofte brugte pakker
\usepackage{amssymb}
\usepackage[danish]{babel}
\usepackage[utf8]{inputenc}
\usepackage{listings}
\usepackage{fancyhdr}
\usepackage{hyperref}
\usepackage{booktabs}
\usepackage{graphicx}
\pagestyle{fancy}
\fancyhead{}
\fancyfoot{}
\rhead{\today}
\rfoot{\thepage}

% Opsæt indlæsning af filer
\lstset{
  language=Python,
  extendedchars=\true,
  inputencoding=utf8,
  linewidth=\textwidth, basicstyle=\small,
  numbers=left, numberstyle=\footnotesize,
  tabsize=2, showstringspaces=false,
  breaklines=true, breakatwhitespace=false,
}

%% Titel og forfatter
\title{Informationsteknologi: Projekt e-læring \\ Projektkursus: Systemudvikling \\Forår 2011}
\author{Arinbjørn Brandsson (hkt789)\\Lasse Ahlbech Madsen (xsc606)\\Naja Mottelson (vsj465)\\Søren Pilgård (vpb984)\\
\\
Gruppeid : LO6\\
\\Instruktor: Lasse Nørregaard}

%% Start dokumentet
\begin{document}

%% Vis titel
\maketitle
\newpage

%% Vis indholdsfortegnelse
\tableofcontents
\newpage

%% HER STARTER RAPPORTEN
\section{Erfaringer med evalueringer af IT-løsningen i samarbejde med brugerne}
Denne sektion er en smule besværlig for vores gruppe at skrive, eftersom vi på nuværende
tidspunkt ikke har en kørende prototype af vores program som kan afprøves af brugerne. 
Som beskrevet i de hidtidige rapporter kommer vores program til at bede brugeren om at kode
et bestemt spil - siden Delrapport 2 har vi brugt vores tid på at implementere dette spil.
Vores motivation til at prioritere på denne måde er at det (fra vores synspunkt) vil være
umuligt at implementere de enkelte trin i undervisningsforløbene hvis man ikke har et 
solidt kendskab til spillets tiltænkte struktur.  

\subsection{Problematikker ifbm. afprøvning}
Dette er en problematisk situation: eftersom vores program beskæftiger sig med pædagogik
er netop afprøvning med brugere en central del af udviklingen. Dog: i forbindelse med dette 
projekt er vores spil nærmere at betragte som data end som 
programmel, og det er ikke selve spillet der skal afprøves af brugerne.

I vores arbejde har vi fundet frem til to essentielle udfordringer mht. afprøvning, som præger 
vores projekt i en smule højere grad end normalt. Først og fremmest vurderer vi at det potentielt 
er sværere for os at lave traditionel løbende afprøvning, eftersom vores 
program ikke rigtig lader sig opdele og afprøve separat. Det er derfor en smule besværligt at 
sikre en optimal iterativ afprøvning i det hele taget. 

Mere vigtigt end dette er dog hvad man kan kalde forudsætningerne for vores program - selve den
måde interaktionen er planlagt er simpelthen problematisk at teste, eftersom en typisk use case
vil vare betragteligt længere end fx. use casen for et af de databaseadministrationssystemer der
præsenteres i lærebogen. Et læringsværktøj er jo netop et program man forventes at bruge som
minimum nogle timer på at arbejde med, og dette har vi ikke resurser til at afprøve i gruppen. 
En anden, mere praktisk del af denne problematik er at vores tilgængelige brugergruppe muligvis er 
uegnet til at give et repræsentativt resultat. Vores ønske med programmet er at skabe et læringsværktøj
som kan benyttes af folk på et forholdsvis lavt niveau af programmeringserfaring, men det kræver
dog kendskab til bl.a. grundlæggende python-syntaks (en tillæringsopgave som ikke er triviel for
mange førstegangsprogrammører). Den datalogiklasse vi samarbejder med har ikke modtaget undervisning
i python, så de eneste af dem vi ville kunne benytte som testpersoner ville være dem der kunne 
overtales til at sætte sig ind i grundlæggende python på egen hånd, eksternt fra undervisningen. 
De elever der ville være villige (og i stand til) dette er sandsynligvis mere engagerede og dygtigere
end normen, og ville således ikke give et helt troværdigt indblik i de spørgsmål vi kunne tænke os
at få besvaret. 

Med baggrund i disse overvejelser har vi derfor besluttet at følge en lidt anderledes afprøvningsstrategi,
som dokumenteres i dette afsnit. 

\subsection{Overvejelser vedr. brugertests}
Som det kan ses af billeddokumentationen i sidste rapport, kommer den grafiske brugergrænseflade
for vores program til at være meget simpel. Endvidere føler vi selv at vi har et godt
billede af hvordan den skal udformes fra papir-mockup-testingen som beskrives i Delrapport 2.
Vi mener derfor ikke at det GUI-specifikke egentlig er af største vigtighed når det kommer til bruger-afprøvning. 
Hvad der derimod er kritisk at afprøve er udformningen af de forskellige trin vi har opdelt 
kodeopgaverne i - om niveauet er for højt/lavt, om opgavebeskrivelserne fungerer, om rækkefølgen
er intuitiv mm. 

\subsection{Planlægning af brugertests}
I vores kommende afprøvningsarbejde er vi, som nævnt, nødt til at overvinde den forhindring at en
kørende prototype mangler. Måden vi regner med
at gøre dette kan siges at være en variation over sidste rapports mockup-testing.

- Vi tegner og fortæller i stedet for at give dem software 
- Vi afprøver med maks tre elever, afhængigt af hvor mange der er interesserede. 
- Spørgsmålene vi vil stlle planlægges at være på formen 'Beskriv hvordan du ville gøre x'
- Måske er der endda fordele ved denne måde at gøre det på: Vi kommer udenom python-problematikken. 		



\end{document}


