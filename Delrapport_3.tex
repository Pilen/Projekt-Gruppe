\documentclass[10pt,a4paper,danish]{article}
%% Indlæs ofte brugte pakker
\usepackage{amssymb}
\usepackage[danish]{babel}
\usepackage[utf8]{inputenc}
\usepackage{listings}
\usepackage{fancyhdr}
\usepackage{hyperref}
\usepackage{booktabs}
\usepackage{graphicx}
\pagestyle{fancy}
\fancyhead{}
\fancyfoot{}
\rhead{\today}
\rfoot{\thepage}

% Opsæt indlæsning af filer
\lstset{
  language=Python,
  extendedchars=\true,
  inputencoding=utf8,
  linewidth=\textwidth, basicstyle=\small,
  numbers=left, numberstyle=\footnotesize,
  tabsize=2, showstringspaces=false,
  breaklines=true, breakatwhitespace=false,
}

%% Titel og forfatter
\title{Informationsteknologi: Projekt e-læring \\ Projektkursus: Systemudvikling \\Forår 2011}
\author{Arinbjørn Brandsson (hkt789)\\Lasse Ahlbech Madsen (xsc606)\\Naja Mottelson (vsj465)\\Søren Pilgård (vpb984)\\
\\
Gruppeid : LO6\\
\\Instruktor: Lasse Nørregaard}

%% Start dokumentet
\begin{document}

%% Vis titel
\maketitle
\newpage

%% Vis indholdsfortegnelse
\tableofcontents
\newpage

%% HER STARTER RAPPORTEN
\section{Erfaringer med evalueringer af IT-løsningen i samarbejde med brugerne}
Denne sektion er en smule besværlig for vores gruppe at skrive, eftersom vi på nuværende
tidspunkt ikke har en kørende prototype af vores program som kan afprøves af brugerne. 
Som beskrevet i de hidtidige rapporter kommer vores program til at bede brugeren om at kode
et bestemt spil - siden Delrapport 2 har vi brugt vores tid på at implementere dette spil.
Vores motivation til at prioritere på denne måde er at det (fra vores synspunkt) vil være
umuligt at implementere de enkelte trin i undervisningsforløbene hvis man ikke har et 
solidt kendskab til spillets tiltænkte struktur.  

\subsection{Problematikker ifbm. afprøvning}
Dette er en problematisk situation: eftersom vores program beskæftiger sig med pædagogik
er netop afprøvning med brugere en central del af udviklingen. Dog: i forbindelse med dette 
projekt er vores spil nærmere at betragte som data end som 
programmel, og det er ikke selve spillet der skal afprøves af brugerne.

I vores arbejde har vi fundet frem til to essentielle udfordringer mht. afprøvning, som præger 
vores projekt i en smule højere grad end normalt. Først og fremmest vurderer vi at det potentielt 
er sværere for os at lave traditionel løbende afprøvning, eftersom vores 
program ikke rigtig lader sig opdele og afprøve separat. Det er derfor en smule besværligt at 
sikre en optimal iterativ afprøvning i det hele taget. 

Mere vigtigt end dette er dog hvad man kan kalde forudsætningerne for vores program - selve den
måde interaktionen er planlagt er simpelthen problematisk at teste, eftersom en typisk use case
vil vare betragteligt længere end fx. use casen for et af de databaseadministrationssystemer der
præsenteres i lærebogen. Et læringsværktøj er jo netop et program man forventes at bruge som
minimum nogle timer på at arbejde med, og dette har vi ikke resurser til at afprøve i gruppen. 
En anden, mere praktisk del af denne problematik er at vores tilgængelige brugergruppe muligvis er 
uegnet til at give et repræsentativt resultat. Vores ønske med programmet er at skabe et læringsværktøj
som kan benyttes af folk på et forholdsvis lavt niveau af programmeringserfaring, men det kræver
dog kendskab til bl.a. grundlæggende python-syntaks (en tillæringsopgave som ikke er triviel for
mange førstegangsprogrammører). Den datalogiklasse vi samarbejder med har ikke modtaget undervisning
i python, så de eneste af dem vi ville kunne benytte som testpersoner ville være dem der kunne 
overtales til at sætte sig ind i grundlæggende python på egen hånd, eksternt fra undervisningen. 
De elever der ville være villige (og i stand til) dette er sandsynligvis mere engagerede og dygtigere
end normen, og ville således ikke give et helt troværdigt indblik i de spørgsmål vi kunne tænke os
at få besvaret. 

Med baggrund i disse overvejelser har vi derfor besluttet at følge en lidt anderledes afprøvningsstrategi,
som dokumenteres i dette afsnit. 

\subsection{Overvejelser vedr. brugertests}
Som det kan ses af billeddokumentationen i sidste rapport, kommer den grafiske brugergrænseflade
for vores program til at være meget simpel. Endvidere føler vi selv at vi har et godt
billede af hvordan den skal udformes fra papir-mockup-testingen som beskrives i Delrapport 2.
Vi mener derfor ikke at det GUI-specifikke egentlig er af største vigtighed når det kommer til bruger-afprøvning. 
Hvad der derimod er kritisk at afprøve er udformningen af de forskellige trin vi har opdelt 
kodeopgaverne i - om niveauet er for højt/lavt, om opgavebeskrivelserne fungerer, om rækkefølgen
er intuitiv mm. 

\subsection{Planlægning af brugertests}
I vores kommende afprøvningsarbejde er vi, som nævnt, nødt til at overvinde den forhindring at en
kørende prototype mangler. Måden vi regner med
at gøre dette kan siges at være en variation over sidste rapports mockup-testing.

- Vi tegner og fortæller i stedet for at give dem software 
- Vi afprøver med maks tre elever, afhængigt af hvor mange der er interesserede. 
- Spørgsmålene vi vil stlle planlægges at være på formen 'Beskriv hvordan du ville gøre x'
- Måske er der endda fordele ved denne måde at gøre det på: Vi kommer udenom python-problematikken. 		

\section{Projektsamarbejdet}
Siden sidste aflevering har vi holdt pause i påskeferien, og har derfor kun haft to uger til at holde møder og arbejde i. Herefter har vi dog holdt jævnlige møder, hvor vi har uddelegeret, prioriteret og planlagt den videre udvikling af projektet og rapportskrivningen. 

Vi oplever visse kommunikationsproblemer i gruppen. Folk er dårlige til at melde ud om status på deres opgaver, hvilket er aldeles problematisk. Det begrænser vores overblik af den hidtidige udvikling af projektet. Man kan ikke hjælpe dem, hvis de sidder fast, da de ikke melder ud om dette. Er de overhovedet i gang? Er der andre, som bliver nød til at overtage opgaven? Eller er den måske næsten færdig? I sidste ende lægger det pres på resten af gruppen og er med til at skabe interne konflikter. For at løse dette, har vi forsøgt os med alvorlige samtaler, med de medlemmer af gruppen, som opfører sig sådan. Vi håber at det vil være nok, da vi ellers ikke kan se, hvad de skal gøre for at forbedre situationen. Ydermere er der en tendens til at tage lidt for let på tidsfrister. Ting bliver lavet meget sent, og det er svært for den ansvarshavene at nå at læse det igennem og give kritisk input. I sidste ende står vi med et dårligere produkt. For at løse dette har vi prøvet at uddelegere opgaverne tidligere, så folk har længere tid til at lave dem. Det virkede ikke. Som et fremtidigt eksperiment, forsøger vi at uddelegere opgaver parvis. Det vil sige at en en enkelt person får ansvaret for at produktet bliver lavet, mens den anden søger for at følge processen, og holder øje med at den ansvarshavende ikke går død i det.

\subsection{Mødeform og dokumentationsstrategi} 
Til egne møder holder vi os stadig til en ret basal referat strategi, hvor vi noterer beslutninger og statusopdateringer ved de enkelte punkter i dagsordenen og konkrete forslag og tanker ved diskussioner. Dette er praktisk på kort sigt, men det giver ikke en den store indsigt i, hvorfor vi har taget de beslutninger vi har, og kan i værste fald være ikke brugbare i en eksamenssammenhæng, når vi skal redegøre for processen. Dette er stadig et punkt vi forsøger at forbedre, men det er ikke prioriteret særligt højt i forhold til mere nærværende problemer. Rapporterne tager meget tid, og vi er ikke så langt med selve it-projektet, som vi gerne ville være. Konsekvensen af dette er, at vi ikke kan finde tid til at udbedre den mangel, og fortsætter med knap halvdårlige referater. Referaterne deler vi over Github (sammen med vores kode), og er derved let tilgængelige for alle gruppemedlemmer. Dette er vi glade for, om end vi har problemer med, at det ikke er alle, der har lige nemt ved at finde ud af git.

Vores møder foregår som regel med stramme dagsordener, der er med til at effektivisere dem, og minimere spildtid. Vi udpeger en referent, og går punktvist frem. Der er for det meste fuldt fremmøde, men vi spilder en betragtelig mængde tid på, at folk kommer for sent. For at udbedre situation forsøger vi at lægge møderne i forlængelse af forelæsninger og øvelsestimer, så folk allerede er her. Som forlængelse af dette, har vi også problemer med sygdom, og vi mangler ofte en person. Dette er der selvfølgelig ikke en helt masse at gøre ved, men vi forsøger at opfordre folk til overveje, om de nu også har det SÅ dårligt, og måske kunne komme alligevel.


\subsection{Projektstyring og implementation}
Som nævnt i prototypeafsnittet, så har vi ingen kærende prototype. Dette er ikke uproblematisk, men vi har valgt at fokusere på at lave den pædagogiske del af projektet først. Dette har vi gjort, fordi programmet i sig selv er meget simpelt, og ikke rigtig kan testes, ej heller ville det give mening at teste det mere, end vi allerede har gjort med vores papirsprototyper. Programmet drejer sig i høj grad om indholdet af forløbene, og uden nogen forløb er programmet tomt og ubrugeligt. Derfor har vi valgt at udvikle det første forløb først eller evt. en del af det, så det kan testes. Denne del af programmet, kan også håndkøres, og er egnet til testing.

Som nævnt i den tidligere rapport havde vi et problem med, at det meget var en enkelt person, der stod for ansvaret for det hele. Dette var ikke udelukkende problematisk, fordi denne person stod med en væsentlig arbejdsbyrde, men også grundet dataindkapsling. Det var kun den ene person, som havde et egentlig overblik over opgaven. Dele af projektet var der ikke andre, som tog del i, og dermed havde resten af gruppen ikke rigtig noget indblik i det. Dette er stadig et problem i gruppen. Vi har ikke haft tid til at fokusere på omlægning af arbejdsstrukturen, og finder os derfor stadig i samme situation. 

For at øge kontrollen med projektet og sikre at vi får det hele med, har vi uddelt 4 af hovedpunkterne i projektet til en person hver. Disse er projektledelse, kodeudvikling, unit-tests og dokumentation af kode. Tanken er ikke, at hver enkelt person står for sit punkt alene, men vedkommende har ansvaret for, at det bliver gjort og har opsyn med punktets status. På denne måde aktiverer og engagerer vi også alle gruppemedlemmer, så ansvaret ikke hænger på blot en del af gruppen, som det ellers har været præget af tidligere i projektet. Ydermere har vi fordelt de første trin i vores program, så vi hver især skriver udkast til 2-3 af trinene i vores første forløb, dog skal alle kigge på samtlige punkter. Dette er gjort fordi punktet i høj grad omhandler pædagogik, og man kan forestille sig ret forskellige tilgangsvinkler til dette. Ved at splitte det op sikrer vi, at vi får alles forslag til formidlingen, og kan i fællesskab finde en optimal løsning, hvor vi bruger de bedste dele fra de enkelte vinkler. En anden vigtig grund er selvfølgelig ansvarsfordeling, hvor vi sørger for at alle laver noget, og det ikke bliver en delmængde af gruppen, som kommer til at stå for det hele.  

Vi er kommet ordentligt i gang med selve projektet nu, der har ellers hængt lidt, og blevet nedprioriteret i forhold til rapportskrivningen. På trods af begrænset tid, går det relativt hurtigt fremad, og vi regner med at kunne have i hvert fald en kørende prototype ved kursets slut, om end ikke helt så omfattende, som vi kunne have ønsket. 

\end{document}


