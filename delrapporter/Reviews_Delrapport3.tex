\documentclass[10pt,a4paper,danish]{article}
%%Pakker
\usepackage{amssymb}
\usepackage[danish]{babel}
\usepackage[utf8]{inputenc}
\usepackage{listings}
\usepackage{fancyhdr}
\usepackage{hyperref}
\usepackage{booktabs}
\usepackage{graphicx}
\pagestyle{fancy}
\fancyhead{}
\fancyfoot{}
\rhead{\today}
\rfoot{\thepage}
%%BeginDocument
\begin{document}

\tableofcontents
\newpage

\section{Resumé}
\subsection{User Testing, Discount User Testing, af Rolf Molich}
Artiklen handler om usability testing, (User testing), hvordan man holder en vellykket brugertest, samt hvordan man sørger for at sin testbruger føler sig godt tilpas før, under og efter en brugertest, samt forskellige teknikker man kan tage i brug. Artiklen beskriver for det meste en test procedure, kaldet "Thinking Aloud Testing", hvor testbrugeren siger højt hvad det er han tænker mens han bruger programmet, (ting som hvad han er i tvivl om, hvordan han forventer at programmet reagerer, og hvordan han fortolker en besked). Disse ting bliver enten skrevet ned af en tester, eller optaget på video eller på lydbånd.

En "Thinking Aloud Testing" kræver dog forberedelse. Først skal man tjekke om testbrugerne er over- eller under-kvalificeret ved at interviewe dem på forhånd, (dog dette behøves ikke for vores brugertest, da vi låner nogle gymnasie elever som læser datalogi til at teste programmet siden de ligger i vores målgruppe). Siden skal man også lave en liste af opgaver som testbrugerne skal løse, hvor den første opgave skal være en nem opgave som kan løses på få minutter, så at testbrugerne ikke bliver stresset eller nervøse. Denne liste af opgaver er en integreret del af vores hjælpeprogram, og derfor vil en del af vores brugertest gå ud på at se om hvis testbrugerne kan forstå disse opgaver. Det er også en god idé at afslutte en brugertest med en debriefing, hvor man taler med testbrugeren, og finder ud af hvilken tanker de havde om programmet, hvad var godt ved programmet, og hvad man kunne gøre bedre eller formulere bedre. En vigtig del under debriefing er at man behandler testbrugernes foreslag med respekt, for ellers vil de være tilbageholdende med foreslag og meninger omkring projektet.

Artiklen gennemgik også kortvarigt en anden test procedure, "Constructive Interaction", hvor man lod to testbrugere arbejde sammen for at kunne løse opgaverne, hvorefter de vil diskutere programmet, hvad det gjorde godt, og hvad kan laves om. Vi vælger dog at ikke tage denne test procedure i brug, eftersom vores testbrugere er gymnasieelever vil de sikkert bruge tiden ukonstruktivt, og ikke tage deres opgave seriøst.

Brugertest er afgørende for vores projekts success. Da vores program er et hjælpeprogram, designet til at lære gymnasieelever til at kunne kode spil i python, skal de instruktioner og opgaver som programmet viser frem til brugeren være klare og nemme at forstå. Derfor passer "Thinking Aloud Testing" rigtig godt til vores projekt - ved brug af denne teknik kan vi, fra vores kunder, høre hvad det er de forventer programmet skal kunne gøre, samt om hvis nogle af vores hints, opgavespecifikationer eller meddelser er uklare, eller ikke fortæller nok om opgaven. Artiklen gav også tips til god etikette før, under og efter en brugetest, hvilket kommer til at være meget nyttige for os, da vores testbrugers input er uhyre vigtigt for os.

Ikke kun det, artiklen vil også hjælpe os med at skrive de trin som vores fremtidige brugere skal gennemføre - artiklen gav gode råd som, for eksempelt, det første trin skal være det nemmeste så at testbrugerne ikke ville føle sig frustreret eller gå i panik hvis de ikke kan løse det første trin. Man kan vel sige at artiklen gav os gode pædagogik råd, hvilket er præcist det som vi har brug for til vores projekt. Derfor kan man sige at artiklen gavner vores gruppe både før vores brugertest, samt efter vi har holdt vores brugertest.

\subsection{Software Architecture in Practice, af Bass, Clements og Kazman}
Artiklen handler om software architecture, hvor software architecture er systemets struktur, eller strukturer, som omfatter sofware elementer, og de relationer imellem elementerne, samt de antagelser som elementer kan have om andre elementer. Arkitekturen definere softwarets elementer, og omfatter de offentlige dele af softwaret. Et arkitektur består af strukturer, (dette plejer at være i flertal medmindre softwaret er meget specialiseret eller simpelt), og disse strukturer behøver ikke at være af samme type, da strukturerne forventes at dække over alle dele af systemet.

Strukturerne for vores program regnes at være af én slags - en modulær baseret struktur, (hvor systemet kan splittes i mindre dele), da dette passer bedre for vores projekt. Siden at hovedparten af vores projekt består af trin, som består af mindre trin, kan denne form af struktur passe godt til systemet, siden at dette lader systemet blive brudt ned i mindre dele. Og siden hvert del af disse trin behøver et overordnet trin for at kunne udføres, (et trin har 'checkpoints', som derefter har hints), sørger denne slags struktur for at systemet bliver sådan.

Derfor kan det være at det bedste form for struktur for vores system ville være at bruge et 'logical' struktur, hvilket opfylder det basis som udgør et modulært struktur. Desværre gav artiklen ikke en bedre beskrivelse af hvad et logical struktur er, udover at det er modulært, og består af objekter, så det kan godt være at det ikke er lige så perfekt som vi tror det er. Men ud fra hvad artiklen har fortalt, virker det godt for os.

De andre former for strukturer som artiklen fortalte om virkede som om de ikke ville fungere for vores system, og derfor vil vores system kun have et struktur. Man kan måske bruge et 'physical' struktur, men da vi lader et tredje-parti program processere koden, samt vi ikke rigtig har brug for meget kommunikation imellem elementerne, kan det nok være at det bare ville komplicere ting at have det med.

Da vi, i gruppen, har haft problemer med at kommunikere med hinanden om hvor vi står med vores arbejde, hvad vi skal gøre med vores dele af rapporten, m.m., vil et arkitektur hjælpe os som en gruppe, da et arkitektur vil stå som et basismodel for det system som vi skal lave. Ved at have en basismodel stående fremme, vil vi alle samme vide hvor i projektet vi står, og det vil gør det nemmere for os holde os organiseret - noget som gavner alle grupper. Og hvis der opstår uenigheder eller misforståelser omkring vores opgaver, vil et arkitektur hjælpe os videre.

\subsection{Foundations for the Study of Software Architecture, af Perry, D. og Wolf, A}
Artiklen fortæller om en anderledes form for software architecture end præsenteret. Artiklen påstår at et systems arkitektur består af tre dele: 
\begin{enumerate}
\item Elementer,
\item Form, (Elementernes begrænsninger), og
\item Basis for systemet, (dette plejer at være systemets krav).
\end{enumerate}
Dette er en temmelig stor forskel. Bass, Clements og Kasman påstår at systemets arkitektur består af mange forskellige strukturer, som så kollektivt beskriver systemets arkitektur, hvorimod Perry og Wolf ser arkitektur som tre kriterier som bliver defineret, samt det er forventet at arkitekten arbejder med kunden for at kunne definére systemets arkitektur. 

Dette fører til et problem: Hvilken form for system arkitektur er så bedre at bruge for vores projekt? Umiddelbart vil vi sige at den form for system arkitektur beskrevet i Foundations for the Study of Software Architecture ville være af større gavn for vores system. Hvis vi skulle tegne op alle de strukturer som kan beskrive vores system, ville vi ende med ikke særlig mange, hvilket modarbejder Bass, Clements and Kazmans system arkitekturs styrke - ved at have mange forskellige strukturer, vil arkitekturen kunne beskrive systemet på mange flere måder end hvis man havde meget få strukturer. Dog for mindre projekter, som kræver at man arbejder mere intimt med sin kunde, vil Perry og Wolfs arkitektur form være bedre for vores system.

Noget andet som artiklen diskutere er Process/Data/Connector Interdependence. Dette er navnet som artiklen giver et fænomen hvor når man laver arkitekturen, er det vigtigt at man har forskellige synspunkter i arkitekturen. Artiklen fortæller om at forarbejdning, data og forbindelser alle har en effekt på hinanden, så hvis et synspunkt er fastsat, (for eksempel hvis forarbejdningens synspunkt er givet), vil der blive givet fokus på datastrømmen igennem forarbejdningens elementer. Det som dette betyder er at arkitekturen skal kunne give et overblik på alle dele af systemet, samt at hvis man kigger på den del af systemet, skal man kunne se hvordan den relatere sig til andre dele af systemet. 

Siden en stor del af vores program er datastrømmen imellem vores hjælpeprogram og det text editor som brugeren kommer til at vælge, samt forarbejdningen af disse dataer, er dette punkt meget hjælpsomt for os. Ved at få et overblik over hvordan de forskellige dele af vores program virker, vil vi nemmere kunne se, via vores system arkitektur, hvilken dele af systemet er afhængige på hvilken andre dele af vores system.
\end{document}
