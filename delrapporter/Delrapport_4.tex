\documentclass[10pt,a4paper,danish]{article}
%% Indlæs ofte brugte pakker
\usepackage{amssymb}
\usepackage[danish]{babel}
\usepackage[utf8]{inputenc}
\usepackage{listings}
\usepackage{fancyhdr}
\usepackage{hyperref}
\usepackage{booktabs}
\usepackage{graphicx}
\pagestyle{fancy}
\fancyhead{}
\fancyfoot{}
\rhead{\today}
\rfoot{\thepage}

% Opsæt indlæsning af filer
\lstset{
  language=Python,
  extendedchars=\true,
  inputencoding=utf8,
  linewidth=\textwidth, basicstyle=\small,
  numbers=left, numberstyle=\footnotesize,
  tabsize=2, showstringspaces=false,
  breaklines=true, breakatwhitespace=false,
}

%% Titel og forfatter
\title{Informationsteknologi: Projekt e-læring \\ Projektkursus: Systemudvikling \\Forår 2011}
\author{Arinbjørn Brandsson (hkt789)\\Lasse Ahlbech Madsen (xsc606)\\Naja Mottelson (vsj465)\\Søren Pilgård (vpb984)\\
\\
Gruppeid : LO6\\
\\Instruktor: Lasse Nørregaard}

%% Start dokumentet
\begin{document}

%% Vis titel
\maketitle
\newpage

%% Vis indholdsfortegnelse
\tableofcontents
\newpage

%% HER STARTER RAPPORTEN


\section{Modelkomponent og Funktionskomponent}
\subsection{Punktet giver ikke mening for os}
At snakke om komponentstruktur i vores programmel giver ikke rigtig mening.
Brugeren kan skifte trin, hvilket får programmet til at køre nogle tests eller
man kan bede om flere hints. Så reelt set har vi et modelkomponent, der holder
styr på hvilket trin brugeren er ved og hvilke hints, der er vist. Og et
funktionskomponent, der kan skife trin, vise hints og signalere til model-
komponentet om disse.

Vores stykke software handler mere om dets data, end om dets selvstændige
funktionalitet, og der er ikke så frygteligt meget at sige om det.


\subsection{Modelkomponentet}
Vores modelkomponenet holder styr på hvor langt brugeren er i forløbet og
hvilke tests, der skal køres på dennes kode. Desuden holder den styr på, hvor
mange hints, der er vist.

\subsection{Funktionskomponent}
Funktionskomponentet er kun relevant for når brugeren skifter trin, når dette
sker kørers der test og det signaleres til modelkomponentet, hvis testene ikke
finder fejl. Ligeledes hvis brugeren ønsker et hint opdateres modelkomponentet,
og et nyt hint vises.

\section{Projektsamarbejdet}
\subsection{Projektstyring}
Siden sidste rapport har langt størstedelen af vores arbejde været rettet imod
fremstilling af trinene til projektet og klargøring af test af disse. Vi har
ikke haft så god tid som vi gerne ville have, og er også blevet presset af, at
vores kundes elever er gået på læseferie. 

(Jeg skal vide hvad der kommer til at stå i test-delen, for at fortsætte hele kunde delen).

Vi udviklede trinene ved først at uddelegere 2-3 trin til hvert enkelt gruppe-
medlem, og derefter satte vi os sammen og strømlinede dem, så de blev ens i
format og tilgang. Dette viste sig at være en rigtig god måde at tvinge alle
til at være kreative, og komme med deres bud på en løsning. Vi har ikke lavet
trin til hele første forløb endnu, men vi har nok til at kunne udføre sigende
tests.

Det har vist sig at sidste rapports ide med at fordele hovedansvaret for
forskellige dele af projektet var lidt for ambitiøs. Vi har slet ikke haft tid
eller overskud til at arbejde på alle delene, og i sidste ende har der slet
ikke været nogen grund til at holde hinanden i tovene på de forskellige områder.
I stedet har vi kørt med den samme model som hidtil, hvor vi har en person med
det overordnede ansvar. Det er stadig ikke optimalt, og vi har reelt kun den ene
person med et egentlig overblik over projektet. Det skaber mere arbejde for den
ansvarlige,og sætter de 3 andre medlemmer værre i en eksamenssituation, når de
skal redegøre for forløbet og de forskellige projektdele.

Ydermere er vi pressede af at 2 af vores medlemmer er en del af DIKU-revyen,
det lægger et stort tidspres på dem, og betyder at en større mængde arbejde
end ellers har måttet flyttes ud på de resterende medlemmer. Vi har fået
udsættelse på rapporten, men det fjerner desværre ikke alt presset. Vi havde
ikke nogen møder i ugen op til revyen, og det var kun halvdelen af gruppen,
som rent faktisk arbejde på projektet (udover test?).

I de møder vi har haft, har vi nok engang haft sløsede referater, men de har
heller ikke været så relevant, da alt vores arbejde og tanker røg ned på papir
i forbindelse med trinene, så der har ikke haft den store relevans. Vi har
siden sidste rapport ikke oplevet problemer med opmøde, ud over at folk stadig
kommer for sent. Ideen med at lægge møderne i forlængelse med undervisningen
var som sådan ikke en dårlig ide, men det er som regel mest praktisk at lægge
møderne tidligere på dagen, og folk har det tilsyneladende bedre med at komme
for sent til disse end undervisning. 

Det går godt med at bruge Github, vi har nu alt vores arbejde delt og sorteret
i mapper og alt er let tilgængeligt. Der er heller ikke længere nogen, som
har problemer med det, og det må siges at have været en succes.

\subsection{Perspektivering}

Vi har gennem forløbet arbejdet en del med at forbedre vores dokumentations-
strategi, til tider fungerede den godt, med aktions-reaktions skemaer for vores
interviews, men til vores møder formåede vi aldrig at få skabt en god standard.
De fleste af vores referater kan ikke bruges til meget mere end at tjekke
egentlige beslutninger. I et fremtidigt projekt ville vi absolut udbedre dette,
og bl.a. få argumenter og begrundelser med for de valg vi traf, så vi i mod-
sætning til nu, kan se hvorfor vi valgte at gøre som gjorde.

Som nævnt tidligere i rapporten er vi ret glade for versionsstyring med Github,
og ikke blot for vores kode, men også for vores rapporter. Vi har dog i dette
projekt ikke draget forfærdelig stor nytte af Github i forbindelse med vores
kode, da det blev sat op for sent, og en person har stået med alt kodearbejdet
indtil videre.

I et fremtidigt projekt vil vi forsøge at være hårdere med deadlines, det har
været et gennemgående problem i gruppen, og har skabt stress og dårligere
produkter. Det er dog svært at rette på, når det er et generelt problem i
gruppen, og vi ikke har adgang til andre konsekvenser end at forlade gruppen.
Det bliver uundgåeligt noget med at forsøge at holde hinanden i ørerne, og
forsøge at holde konstant opsyn med status på forskellige opgaver, hvilket er
generende for alle i gruppen.

Selvom vi har været glade for at have en gruppestyrer, så er det ikke sikkert,
at vi vil gentage dette i samme grad. Det er fint at have en overordnet
koordinator, men i små grupper som disse, betyder det blot at vedkommende står
med en større arbejdsbyrde, da der ikke er plads til medlemmer, som ikke laver
ligeså meget, som de andre på projektet. I forlængelse af dette, oplevede vi
også at miste et gruppe medlem midt i forløbet. Det skete meget pludseligt og
uden så meget som et ord til os. Dette betød at vi måtte skære ned for for-
ventninger, til vores endelig produkt og selvfølgelig mere arbejde til de indi-
viduelle medlemmer.

Vi kunne godt have brugt mere kontakt med kunden, det blev meget "vores" projekt
i stedet for et vi lavede i fællesskab. Det virkede heller ikke som om, at vi 
havde den samme forståelse af meningen med projektet, og det ønskede produkt.
Dette kunne bestemt gøres bedre, og ville forventeligt være anderledes i en
situation, hvor vi var hyret til en opgave, i stedet for at vi er ude og sælge
et projekt, som vi godt kunne tænke os at lave. Hvis der var en kontrakt invol-
veret, ville det nok også have været anderledes, da kunden ville være mere 
opsøgende i forhold til status og den videre udvikling af projektet, i stedet
for passivt at vente på, hvad vi nu finder på at aflevere. Det var en risiko vi
tog, da vi valgte at arbejde sammen med en enkelt gymnasielærer ide stressede
månederne op til gymnasieeksamenerne. Dette skulle vi måske have været mere
bevidste om og fundet en anden partner.

Alt i alt er det der generer os mest, at vi har set os nød til at fokusere mere
på at skrive rapporter, end at udvikle et produkt vi kunne være tilfreds med.
Her et par uger før eksamen står vi ikke med noget der ligner et færdigt program,
og det er ikke, fordi det er et voldsomt stort eller kompliceret program, men
fordi rapporterne syntes at fylde mere end udviklingen, og når vi har andre fag
ved siden af, så må man jo prioritere, og et godt produkt giver os ikke en god
karakter i sidste ende, mens et "godt projektforløb" er guld. Vi syntes det er
ærgerligt, at det skal være sådan, men vi har indrettet os efter det, og håber
at det vil være anderledes i fremtidige projekter.

\end{document}
