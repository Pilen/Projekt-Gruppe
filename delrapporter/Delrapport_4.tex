\documentclass[10pt,a4paper,danish]{article}
%% Indlæs ofte brugte pakker
\usepackage{amssymb}
\usepackage[danish]{babel}
\usepackage[utf8]{inputenc}
\usepackage{listings}
\usepackage{fancyhdr}
\usepackage{hyperref}
\usepackage{booktabs}
\usepackage{graphicx}
\pagestyle{fancy}
\fancyhead{}
\fancyfoot{}
\rhead{\today}
\rfoot{\thepage}

% Opsæt indlæsning af filer
\lstset{
  language=Python,
  extendedchars=\true,
  inputencoding=utf8,
  linewidth=\textwidth, basicstyle=\small,
  numbers=left, numberstyle=\footnotesize,
  tabsize=2, showstringspaces=false,
  breaklines=true, breakatwhitespace=false,
}

%% Titel og forfatter
\title{Informationsteknologi: Projekt e-læring \\ Projektkursus: Systemudvikling \\Forår 2011}
\author{Arinbjørn Brandsson (hkt789)\\Lasse Ahlbech Madsen (xsc606)\\Naja Mottelson (vsj465)\\Søren Pilgård (vpb984)\\
\\
Gruppeid : LO6\\
\\Instruktor: Lasse Nørregaard}

%% Start dokumentet
\begin{document}

%% Vis titel
\maketitle
\newpage

%% Vis indholdsfortegnelse
\tableofcontents
\newpage

%% HER STARTER RAPPORTEN


\section{Modelkomponent og Funktionskomponent}

Som jeg ser det, så er det meget begrænset hvad vi har her. Reelt set består
modelkomponentet bare af at brugeren kan se hjælp + hint og evt. nogle beskeder
hvis han allerede har teste det. 

I baggrunden er der et funktionskomponent, som kører unit tests, printer nyt
hjælp og hints hvis brugeren skifter trin og evt. giver nogle fejl meddelelser. 

Der står at vi skal beskrive et klassediagram for det, og det har vi allerede
lavet i en tidligere rapport. Det kan skrives i løbet af 10 minutter, hvis min
forståelse af punktet er korrekt. 

\section{Projektsamarbejdet}
\subsection{Projektstyring}
Siden sidste rapport har langt størstedelen af vores arbejde været rettet imod
fremstilling af trinene til projektet og klargøring af test af disse. Vi har
ikke haft så god tid som vi gerne ville have, og er også blevet presset af, at
vores kundes elever er gået på læseferie. 

(Jeg skal vide hvad der kommer til at stå i test-delen, for at fortsætte hele kunde delen).

Vi udviklede trinene ved først at uddelegere 2-3 trin til hvert enkelt gruppe-
medlem, og derefter satte vi os sammen og strømlinede dem, så de blev ens i
format og tilgang. Dette viste sig at være en rigtig god måde at tvinge alle
til at være kreative, og komme med deres bud på en løsning. Vi har ikke lavet
trin til hele første forløb endnu, men vi har nok til at kunne udføre sigende
tests.

Det har vist sig at sidste rapports ide med at give hovedansvaret for
forskellige dele af projektet var lidt for ambitiøs. Vi har slet ikke haft tid
eller overskud til at arbejde på alle delene, og sidste ende ikke været nogen
grund til at holde hinanden i tovene på de forskellige områder. I stedet har vi
kørt med den samme model som hidtil, hvor vi har en person med det overordnede
ansvar. Det er stadig ikke optimalt, og vi har reelt kun den ene person med et
egentlig overblik over projektet. Det skaber mere arbejde for den ansvarlige,
og sætter de 3 andre medlemmer værre i en eksamenssituation.

Ydermere er vi pressede af at 2 af vores medlemmer er en del af DIKU-revyen,
det lægger et stort tidspres på dem, og betyder at en større mængde arbejde
end ellers har måttet flyttes ud på de resterende medlemmer. Vi har fået
udsættelse på rapporten, men det fjerner desværre ikke alt presset. Vi havde
ikke nogen møder i ugen op til revyen, og det var kun halvdelen af gruppen,
som rent faktisk arbejde på projektet (udover test?).

I de møder vi har haft, har vi nok engang haft sløsede referater, men de har
heller ikke været så relevant, da alt vores arbejde og tanker røg ned på papir
i forbindelse med trinene, så der har ikke haft den store relevans. Vi har
siden sidste rapport ikke oplevet problemer med opmøde, ud over at folk stadig
kommer for sent. Ideen med at lægge møderne i forlængelse med undervisningen
var som sådan ikke en dårlig ide, men det er som regel mest praktisk at lægge
møderne tidligere på dagen, og folk har det tilsyneladende bedre med at komme
for sent til disse end undervisning. 

Det går godt med at bruge Github, vi har nu alt vores arbejde delt og sorteret
i mapper og alt er let tilgængeligt. Der er heller ikke længere nogen, som
har problemer med det, og det må siges at have været en succes.

\subsection{Perspektivering}

"Nøglespørgsmål er bl.a.: Hvorledes har I prioriteret og styret projektindsatsen?
Hvordan har I sikret fremdrift på de felter som var mest kritiske/afgørende for et
succesfuldt resultat? Hvad gik godt? Hvad gik mindre godt? Hvad har I lært, som
måske kan gøres bedre i et kommende IT-projekt?"

Noget spice til dette punkt ville være lækkert, udmiddelbart kan jeg kunne komme i tanke om nedenstående:

Versionsstyring godt
Gruppestyring knap så heldig
Hårde deadlines jatak
Mere og bedre intern kommunikation
Mere kundekontakt?
En mere fladstruktur i gruppen fra begyndelsen af
...


\end{document}
